\section*{Conclusion}
\addcontentsline{toc}{section}{Conclusion}

% Une page tout au plus, résumé du travail accompli, faire apparaître si les objectifs ont été atteints, si de nouvelles difficultés ont été soulevées, propositions de solution et futur développement.

La gestion des comptes à privilèges, qui ne représentait pas un intérêt énorme il y'a encore quelques années, est aujourd'hui un des piliers centraux de la sécurité des systèmes d'information.\\
Cette intérêt pour ces comptes vient majoritairement du fait que 80\% des failles de sécurité impliquent des \glspl{credential} privilégiés selon une estimation de Forrester \cite{acs}. A cette volonté de combler des failles de sécurité s'ajoute la volonté de valider les audits de sécurité en consultant les activités des comptes à privilèges.\\
Ce stage m'aura permis de répondre à ce besoin, avec des solutions d'éditeurs, sur lesquelles il ne faut pas entièrement se reposer. Il est important de poser des règles de sécurité à faire respecter pour chacun, afin de limiter les risques, car le risque zéro n'existe jamais. Il est ausis important de noter que ces solutions devront évoluer avec les systèmes d'informations, qui tendent à être de plus en plus délocalisés (dans le cloud), ou bien à réfléchir à la nécessité pour une entreprise de délocaliser son infrastructure, afin de trouver un équilibre entre sécurité et facilité d'utilisation et d'accès.\\
Ce stage m'aura par ailleurs enrichi au niveau technique en découvrant et me formant sur de nouvelles technologies tout autant qu'au niveau humain, avec la découverte du travail en équipe dans le monde du service informatique. Il a par la même occasion parachevé ma formation, en mettant en application mes compétences acquises au cours de ma formation.\\
J'ai pu découvrir le monde de la sécurité, et la quantité de domaines qu'il comprenait, ce qui me pousse à continuer dans cette voie afin de pouvoir devenir spécialiste dans ce domaine.