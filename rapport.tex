\documentclass[a4paper,11pt,final]{article}
% Pour une impression recto verso, utilisez plutôt ce documentclass :
%\documentclass[a4paper,11pt,twoside,final]{article}

\usepackage[english,francais]{babel}
\usepackage[utf8]{inputenc}
\usepackage[T1]{fontenc}
\usepackage[pdftex]{graphicx}
\usepackage{tikz}
\usepackage{setspace}
\usepackage{hyperref}
\usepackage[toc]{glossaries}
\usepackage[french]{varioref}
\usepackage{enumitem}

\makeglossaries

\newlist{arrowlist}{itemize}{1}
\setlist[arrowlist]{label=$\rightarrow$}

\newcommand{\reporttitle}{État de l'art des systèmes de gestion des comptes à privilèges}     % Titre
\newcommand{\reportauthor}{Alexandre \textsc{Kervadec}} % Auteur
\newcommand{\reportsubject}{Mémoire de stage de fin d'études} % Sujet
\newcommand{\HRule}{\rule{\linewidth}{0.5mm}}
\setlength{\parskip}{1ex} % Espace entre les paragraphes

\hypersetup{
    pdftitle={\reporttitle},%
    pdfauthor={\reportauthor},%
    pdfsubject={\reportsubject},%
    pdfkeywords={rapport} {IAM} {PAM} {synetis} {bordeaux} {rennes} {universite} {stage}
}

\begin{document}
  % Inspiré de http://en.wikibooks.org/wiki/LaTeX/Title_Creation

\begin{titlepage}

\begin{center}

\begin{minipage}[t]{0.48\textwidth}
  \begin{flushleft}
    \includegraphics [width=60mm]{images/logo-univ.jpg} \\[0.5cm]
  \end{flushleft}
\end{minipage}
\begin{minipage}[t]{0.48\textwidth}
  \begin{flushright}
    \includegraphics [width=30mm]{images/mini_logo_synetis.png} \\[0.5cm]
  \end{flushright}
\end{minipage} \\[1.5cm]

\textsc{\Large \reportsubject}\\[0.5cm]
\HRule \\[0.4cm]
{\huge \bfseries \reporttitle}\\[0.4cm]
\HRule \\[1.5cm]

\begin{minipage}[t]{0.3\textwidth}
  \begin{flushleft} \large
    \emph{Auteur :}\\
    \reportauthor
  \end{flushleft}
\end{minipage}
\begin{minipage}[t]{0.6\textwidth}
  \begin{flushright} \large
    \emph{Responsables :} \\
    M. Philippe \textsc{Rolland} \\
    M. Damien \textsc{Seiler} \\
    M. Mathieu \textsc{Raffinot}
  \end{flushright}
\end{minipage}

\vfill

{\large 2016}

\end{center}

\end{titlepage}

  \cleardoublepage % Dans le cas du recto verso, ajoute une page blanche si besoin
  \section*{Remerciements}
\addcontentsline{toc}{section}{Remerciements}

Je tiens tout particulièrement à remercier mes accompagnateurs chez \textsc{Synetis} : Damien Seiler et Philippe Rolland, qui m'ont aidé quotidiennement durant ce stage. Je remercie aussi tous les consultants de chez \textsc{Synetis}, qui ont, à des moments ponctuels, su m'enrichir de leur expertise. Il me semble aussi important de remercier Lionel Clément qui m'a suivi et aidé pour ma recherche de stage, lorsque j'avais un retard conséquent par rapport aux dates prédéfinies.
  \cleardoublepage
  \tableofcontents % Table des matières
  \sloppy          % Justification moins stricte : des mots ne dépasseront pas des paragraphes
  \cleardoublepage
  \listoffigures
  \addcontentsline{toc}{section}{Liste des figures}
  \cleardoublepage
  % \newglossaryentry{credential}
{
	name=credential,
	description=est le terme représentant les informations de connexion, comme par exemple un couple login/mot de passe ou une carte à puce}
}
  \newglossaryentry{credential}{name=credential, description={est le terme représentant les informations de connexion, comme par exemple un couple login/mot de passe ou une carte à puce}}
  \newglossaryentry{pam}{name=PAM, description={Privileged Account Management : gestion des comptes à privilèges}}
  \newglossaryentry{bastion}{name=bastion, description={moteur de gestion des connexions aux ressources de la solution de PAM}}
  \newglossaryentry{stratch}{name={from stratch}, description={démarrer un projet depuis rien, à partir de zéro}}
  \printglossaries
  \cleardoublepage
  \section*{Introduction} % Pas de numérotation
\label{intro}
\addcontentsline{toc}{section}{Introduction} % Ajout dans la table des matières

%-> Environnement professionnel et l'entreprise. <-
%
%-> Objectifs du travail personnel et moyens mis en oeuvre pour les atteindre <-
%
%-> Présentation claire du plan adopté poru la suite du corps du mémoire <-
%
%Pas plus de 2 pages.
Le choix de mon entreprise de stage s'est orienté vers \textsc{Synetis}, qui est une porte ouverte vers le monde de la sécurité, dans lequel je souhaite m'orienter malgré ma formation qui n'est pas spécialisée dans ce domaine.\\
\textsc{Synetis} est une PME\footnote{Petite à Moyenne Entreprise} à taille humaine basée dans le centre de Paris. Une agence a été ouverte à Rennes en 2012 lors du recrutement de spécialistes en gestion des identités sur cette zone. Cette agence de Rennes compte 6 consultants et 2 managers (dont le chef d'agence). L'ambiance est très conviviale, tout le monde se connaît, des évènements de cohésion sont régulièrement organisés (afterwork, repas en groupe le vendredi midi). De même, une réunion trimestrielle est organisée sur le siège de Paris, qui regroupe tout le personnel de \textsc{Synetis}, afin de faire un bilan des projets, de rencontrer les nouveaux arrivants, et de partager les nouveaux objectifs visés par l'entreprise.\\
Le travail est réparti par équipe sur les différents projets en cours, chaque consultant pouvant travailler sur plusieurs projets en même temps. On peut distinguer deux types de projet : les gros projets s'étalant sur de longues périodes (plus de six mois) et les projets courts durant au maximum quatre mois. L'entreprise est spécialisée dans la gestion d'identité pour les grands groupes (conseils régionaux, grandes entreprises nationales et internationales), mais commence à développer une branche de pentest (test d'intrusion sur différentes structures, pour de plus amples informations, le guide de Rafay Baloch \cite{rba} est très complet).\\
J'étais intégré comme tous les autres consultants, avec un tuteur travaillant sur des projets correspondants au sujet de mon stage qui me guidait au travers de mes recherches et développements.\\
%Reprendre ce paragraphe
%=======================
Durant ce stage, j'ai travaillé sur les systèmes de gestion de comptes à privilèges.
Les comptes à privilèges sont les comptes avec des droits élevés, comme ceux des comptes \texttt{root} des systèmes d'exploitation Linux/UNIX et \texttt{Administrateur} sur Windows.
Pour ce projet, il était question de comprendre en profondeur les moyens existant et les enjeux de la gestion de comptes à privilèges.
Cette compréhension avancée a permis de comparer l'ensemble des solutions disponibles sur le marché.
Cette comparaison a mené à considérer 2 solutions à mettre en œuvre sur un environnement virtuel de test.
Ce test fut une mise en situation des solutions, car les spécifications des solutions sont un point clef d'un choix d'achat, mais l'ergonomie et la fonctionnalité peut être le critère décisif de ce choix.
Toutes les informations récoltées nous auront finalement aidé à comprendre la façon dont les solutions répondent à la problématique de la gestion des comptes à privilèges, et de déterminer quelle(s) solution(s) répond(ent) le mieux à notre problématique, afin de pouvoir proposer de l'intégration avec de futurs clients.
%=======================

Le plan adopté pour la suite de ce mémoire consiste en une première section décrivant la problématique à résoudre, et les origines de cette problématique. Une deuxième section expliquera les moyens et les méthodes mises en œuvre pour répondre à cette problématique et enfin une dernière partie expliquera les résultats obtenus.
  \cleardoublepage
  %\section{Section}
%
%\subsection{Sous section}
%
%\subsubsection{Sous sous section}
%
%\paragraph{Paragraphe}
%
%Pas plus bas dans la hiérarchie. Minimum 2 éléments par division hiérarchique.
%
%Equilibrer la taille de chaque chapitre (à l'exception de l'introduction et de la conclusion).
%
%Exemple de développement :
%- Problématique
%- Matériels et méthodes
%- Résultats et discussion

%\begin{figure}[!ht]
%    \center
%    \includegraphics[width=0.5\textwidth]{./images/logo-univ.jpg}
%    \caption{logo université, échelle 50\% de la page}
%\end{figure}
\section{Problématique : comment pallier le manque de contrôle des comptes privilégiés}

L'objectif général est de résoudre le problème de manque de contrôle et de monitoring des comptes à privilèges. Afin de parvenir à ce résultat, plusieurs solutions ont été développées par des éditeurs, le but final de ce projet étant de déterminer quelle est, ou quelles sont, la ou les meilleures solutions permettant de résoudre ce problème.\\
Pour répondre à cette problématique, nous allons d'abord expliquer plus en détails ce qu'est un compte à privilèges, puis nous intéresser aux raisons pour lesquelles ces comptes sont des points clefs de la sécurité des systèmes d'information. Ensuite, nous allons expliquer un fonctionnement global et commun aux solutions du marché puis mettre en avant les limitations que peuvent avoir ces solutions de \gls{pam}.

\subsection{Contexte}

\subsubsection{Les comptes à privilèges}
Il semble important de définir clairement la différence entre un compte à privilèges et un compte d'utilisateur classique, et plus précisément les deux catégories de mot de passe qu'ils engendrent :
\begin{description}
	\item [Les mots de passe utilisateur] : basiquement, un mot de passe est un secret qui permet l'utilisation d'un compte. Un compte représente un utilisateur humain et son mot de passe justifie son identité, comme un compte Active Directory\footnote{Système d'annuaire électronique propriétaire de \textsc{Microsoft}, basé sur la norme LDAP (Lightweight Directory Access Protocol).}, qui représente digitalement un humain, et le mot de passe qui justifie l'identité de l'humain qui s'y connecte auprès du système. Ce type de mot de passe est connu par l'utilisateur humain qui est représenté par le compte, le but étant d'avoir un minimum de comptes par entité humaine, l'idéal étant une correspondance bijective\footnote{Un seul utilisateur humain pour un seul compte et vice-versa.}
	\item [Les mots de passe de comptes à privilèges] : ces mots de passe sont des mots de passe liés à un compte qui ne représente pas une entité humaine. Ce compte peut être un compte système comme \texttt{root} sur un système d'exploitation \textsc{Linux} ou un compte de service sur un système d'exploitation \textsc{Windows}. Ces mots de passe ne sont pas forcément fournis à un utilisateurs humain, et même idéalement, ne doivent pas l'être, ainsi, ils peuvent d'être d'une complexité élevée sans avoir un soucis d'apprentissage de ce dernier
\end{description}

Le but de ce stage était de trouver la meilleure solution pour sécuriser les mots de passe de comptes à privilèges tout en supervisant les comptes liés à des utilisateurs.


\subsubsection{La gestion de comptes à privilèges}
La gestion de comptes à privilèges (PAM\footnote{Privileged Access Management}) est une sous-section de la gestion d’identité et d’accès (IAM\footnote{Identity and Access Management}). L’IAM est un large champ de contrôle d’accès qui se veut critique dans le domaine des technologies de l’information.\\
Il existe bien sûr une multitude de connexions spécifiques entre les utilisateurs et les dépendances technologiques. La \gls{pam} n’est que l’une d’entre elles.\\
La \gls{pam} est apparue au début des années 2000 à cause de l’impossibilité des solutions d’IAM de contrôler, gérer et faire des rapports sur les accès aux serveurs, aux bases de données, aux équipements réseau et tout autre application critique au sein d’une organisation. Cette solution entraîne une gestion d’un petit nombre d’utilisateurs, mais d’un grand nombre de dépendances technologiques ayant une importance clef dans le fonctionnement des infrastructures.\\
La fonctionnalité principale d’une solution de gestion d’identité privilégiée gravite autour de la sécurisation de l’accès aux ressources critiques par les administrateurs IT.\\
Plus précisément, des privilèges spécifiques peuvent être délivrés à des comptes d’utilisateur. Ces privilèges sont dépendants des systèmes et des applications impliqués, mais ils peuvent inclure la capacité à écrire des données, créer des comptes, exécuter une mission, pour ne citer que quelques exemples. En plus de contrôler l’accès avec un grand niveau de finesse, beaucoup de solutions fournissent aussi un package d’actions disponibles lors de l’utilisation d’un compte privilégié. Dans le but d’assurer la sécurité de ces comptes, les solutions exploitent un large éventail de mécanismes, comme par exemple, l’utilisation de clefs SSH, la rotation de mot de passe et l’ajout de multiples authentifications (authentification multi-facteur).\\
La gestion d’identités privilégiées est devenue de plus en plus importante, notamment avec la croissance des requis de sécurité et des règles de conformité (l’ISO 27001 qui est une norme de sécurité international sur la protection des actifs informationnels par exemple). Il est aussi important de noter que de nombreux exploits\footnote{Element permettant d'exploiter une faille de sécurité} compromettant des données sont liés à la compromission d’accréditations privilégiées. Avec la recrudescence des tentatives de hack et les contrecoups économiques de plus en plus conséquents, les entreprises commencent à porter de plus en plus d’intérêt à la gestion et au contrôle des comptes à privilèges. En effet, les tentatives de piratage sont de plus en plus variées : du social engineering, au vol de ces accréditations par une brèche dans la sécurité en passant par des \textit{brute force} de mot de passe n’ayant pas une complexité suffisante\footnote{L'utilisation de mots de passe faibles tels que \texttt{password}, \texttt{admin}, \texttt{1234}, \texttt{azerty1234} ou \texttt{Abcd1234} reste encore très fréquente. Pour trouver ces mots de passe faibles, la technique la plus employée est le brute force avec un dictionnaire de mots de passe} (dont traitent Weber \emph{et coll.} \cite{jew}).\\

\subsection{Un point clef de la sécurité des systèmes d'information}

\subsubsection{Des prestataires de service ayant les clefs du royaume}
\label{par:presta_kingdom}

Le principal point de sécurité posant question avec les comptes à privilèges est leurs utilisateurs.\\
Dans une entreprise, le personnel gérant l'infrastructure n'est pas forcément d'une dimension adaptée aux besoins. Cette dimension n'est souvent pas accessible et est comblée par l'embauche de prestataires de service. On peut prendre l'exemple d'une entreprise de taille moyenne, qui gère son infrastructure en autonomie. Seulement, l'entreprise grossit et une nouvelle infrastructure déployée en \emph{clusters}\footnote{Grappe de serveurs sur le réseau, aussi appelé ferme de calcul.} est nécessaire. Le personnel n'ayant pas les compétences pour réaliser cette mise à niveau, l'entreprise devra faire appel à une entreprise de service tierce, qui réalisera et maintiendra cette nouvelle infrastructure. Ce sont les prestataires de service.\\
Cependant, ces prestataires doivent intervenir avec des droits élevés d'administration. Ils ont donc ce qu'on appelle les clefs du royaume\footnote{En anglais, littéralement \emph{keys to the kingdom}, représente l'accès sans limite à toute l'infrastructure informatique.}, sans aucune supervision.\\
Du point de vue de la sécurité, il est très risqué de recourir à de telles pratiques. C'est ici qu'une solution doit être trouvée, et c'est ici que les systèmes de gestion des comptes à privilèges ont un rôle clef.

\subsubsection{Une cible de choix pour les pirates}
Un compte privilégié est un compte utilisé par les administrateurs système et réseau, ainsi que par les équipes de sécurité pour accéder aux ressources réseau comme les serveurs, les pare-feux, les switches, les routeurs, les ordinateurs, les applications ou les bases de données avec des droits élevés. Ces comptes sont nécessaires à la maintenance d'une infrastructure, tout comme aux interventions de réparation, de diagnostique ou gestion de situations de crise\footnote{Attaque sur un serveur par exemple}. Dans de grands groupes, il peut y avoir un grand nombre de ces comptes, de l'ordre de la centaine voire du millier d'entités, répartis sur plusieurs sites.\\
Ces comptes peuvent aussi être des comptes d'application communicant avec d'autres applications\footnote{A2A : Application To Application, littéralement d'application à application}, comme par exemple un serveur faisant une sauvegarde régulière sur un autre serveur de récupération.\\
Tous ces comptes ne sont pas surveillés par les traditionnels gestionnaires d'identités, seul un mot de passe permet d'y accéder. De plus ces comptes sont très souvent des comptes partagés entre plusieurs administrateurs pour une question de facilité de gestion.\\
Ainsi, une personne mal intentionnée réussissant à voler les accès d'un tel compte verrait son pouvoir de destruction, voire de vol d'informations sans limites, ce qui en fait une cible privilégiée par les pirates informatiques.

\subsubsection{Un manque de visibilité d'actions} Comme abordé en introduction, il existe un grand manque de visibilité sur les actions des comptes à privilèges. En effet, ces comptes aux droits élevés, ne sont ni tracés, ni surveillés. Ceci peut avoir plusieurs mauvaises incidences, certaines intentionnelles et malveillantes, d'autre involontaires, mais tout de même paralysantes.\\
On peut classer ces incidences en deux catégories, l'une relevant d'une erreur accidentelle, et l'autre d'une volonté de nuire à une organisation comme le décrit l'article de \textsc{Shackleford} \cite{dsh} :
\begin{itemize}
	\item Erreur accidentelle :
	\begin{arrowlist}
		\item Erreur de configuration, difficile à retrouver à cause du manque de supervision
	\end{arrowlist}
	\item Volonté de nuire :
	\begin{arrowlist}
		\item Sabotage d'une configuration, menant à un déni de service
		\item Vol d'informations sensibles
	\end{arrowlist}
\end{itemize}

Ce manque de  visibilité crée aussi un point noir dans un audit de sécurité : il n'y a aucune détection de faille de sécurité concernant ces comptes.

\subsection{Objectifs d'un système de \gls{pam}}

D'après les points précédents, on peut en déduire les spécificités que nous voudrions améliorer vis-à-vis des comptes à privilèges :
\begin{itemize}
	\item Centraliser l’accès aux données de l’entreprise
 	\item Sécuriser les comptes à privilèges (duo identifiants et mot de passe)
 	\item Gérer de manière forte des mots de passe et établir une politique d’authentification forte
	\item Journaliser et superviser les activités des comptes à privilèges
\end{itemize}

\subsection{Fonctionnement d'un système de gestion de comptes à privilèges}

\subsubsection{Cas général}
Le principe commun à toutes les solutions des éditeurs est la présence d'une identification sur un serveur central. On peut considérer 2 groupes : les comptes à privilèges et les ressources à protéger. Entre ces 2 entités vient s'intercaler le serveur d'identification, qui fait la passerelle entre les comptes et les ressources.\\
L'identification d'un utilisateur sur un compte à privilèges se fait sur le serveur central, et l'identification de ce compte à privilèges sur une ressource protégée est déléguée au serveur central. Ainsi, les utilisateurs n'ont plus à gérer et connaître les mots de passe d'accès aux ressources protégées; c'est le serveur central qui détient tous les secrets\footnote{Les logins et mot de passe, aussi appelés \og \glspl{credential} \fg{} en anglais.}.\\
Le serveur central a, à sa charge, de protéger les mots de passe dans un coffre-fort et de les renouveler régulièrement (il est préconisé de renouveler ses mots de passe au moins une fois par mois, ce qui est déjà peu pour une entreprise).

\begin{figure}[!ht]
    \center
    \includegraphics[width=\textwidth]{./images/Schema_ultra_light_PAM.png}
    \caption{Fonctionnement général d'un système de \gls{pam}}
\end{figure}

\subsubsection{Deux types d'architecture}
Malgré le principe de fonctionnement identique pour quasiment toutes les solutions de \gls{pam}, il existe cependant des différences au niveau de l'architecture de ces dernières solutions, on distingue 2 grandes familles :
\begin{itemize}
	\item Architecture proxy\footnote{Toutes les communications transitent par un point de contrôle.} : les ressources et les comptes d'utilisateurs sont séparés physiquement (ou logiquement\footnote{Redirection des paquet  par port.}), le serveur central gère tout seul les accès aux applications
	\item Architecture avec agents : les accès aux ressources et la supervision sont gérés par des agents sur les ressources cibles (application installée sur la ressource)
\end{itemize}
Nous pouvons rapidement nous rendre compte qu'une architecture avec des agents est beaucoup plus intrusive et longue ou difficile à mettre en place qu'une architecture en proxy, où seul l'adressage est à modifier.

\subsection{Les limitations des solutions \gls{pam}}

\subsubsection{Le facteur humain} Il faut noter que même avec un système de \gls{pam} éprouvé et efficace, nous ne pouvons pas mettre de côté le risque le plus exploitable dans le domaine de la sécurité qu'est, le facteur humain. En effet, il est souvent plus facile de tromper un élément du personnel pour une compromission de données. Il sera donc important lors d'un déploiement d'une solution de \gls{pam}, d'éduquer le personnel de l'organisation concernée, afin que ces derniers mettent correctement en œuvre les règles de base de la sécurité informatique comme par exemple :
\begin{itemize}
	\item Totalement prohiber \og l'effet \textit{post-it} \fg{} : notation de mot de passe sur un \textit{post-it} collé sur l'écran
	\item Toujours remplacer les mots de passe d'usine dans les logiciels utilisés
	\item Forcer le changement régulier (au minimum une fois par mois) de mot de passe des utilisateurs
\end{itemize}

\subsubsection{La détection de comportements anormaux}
Très peu de solutions de \gls{pam} proposent un système de détection de comportements anormaux, comme par exemple un conseiller commercial qui aurait accès aux informations des comptes clients, qui abuserait de ce droit.\\
En effet, même avec une restriction des droits, une supervision et journalisation des activités, une solution de \gls{pam} n'est pas une intelligence artificielle qui peut détecter des comportements suspects. Cependant, il est toujours possible de détecter des comportements prédéfinis avec des successions de commandes qui lèveraient une alerte.

%L’objectif est de mettre en place un système de contrôle des risques liés aux comptes à privilèges, donc tout d’abord avoir un socle central d’accès. Ce socle doit constituer l’élément central du système d'information par lequel circulent tous les flux venant des utilisateurs vers les données de l’entreprise. Ainsi, ce socle gère tous les accès aux ressources protégées, et les utilisateurs n’ont plus à connaître les mots de passe des ressources, mais juste leur mot de passe personnel pour se connecter au socle principal.\\
%Le choix d’une solution doit prendre en compte l’environnement technique, les habitudes des administrateurs et les logiciels utilisés par l’entreprise.\\
%Afin de parvenir à un système sûr et performant, nous pouvons dégager les objectifs suivants :
%\begin{itemize}
%	\item Centraliser l’accès aux données de l’entreprise.
%	\item Sécuriser les comptes à privilèges.
%	\item Gérer de manière forte des mots de passe et établir une politique d’authentification forte.
%	\item Journaliser les activités des comptes à privilèges.
%	\item Automatiser les processus pour améliorer la productivité tout en renforçant la sécurité.
%\end{itemize}
  \cleardoublepage
  \section{Méthodes et moyens mis en œuvre}
\label{sec:meth_moy}

Comme le titre l'indique, nous allons expliquer les méthodes et les moyens utilisés pour répondre à la problématique posée. Cette section sera séparée en 3 sous-sections : la première décrivant mon cheminement en matière de gestion de projet, la deuxième traitant de la phase de recherche d'informations, avec ses problèmes rencontrés et solutions trouvée, puis une troisième partie décrivant le travail effectué pendant la phase de test des deux produits sur un environnement de test, ainsi que toutes les différentes technologies utilisées.

\subsection{Gestion de projet}
\label{subsec:gestion}

La première chose que j'ai dû faire en arrivant dans l'entreprise, fut ce qu'on appelle chez \textsc{Synetis} une note de cadrage. Cette note de cadrage correspond, par rapport à ce qu'on a pu faire à l'université durant divers projets, à l'analyse des besoins et les résultats attendus, l'organisation en tâches de l'intégrité du stage. Cette répartition des tâches dans le temps a permis de scinder l'ensemble du projet en de multiples étapes simples, courte, qui m'a permis de segmenter mon stage pour ne pas me retrouver perdu ou submergé par le travail. En dernière partie furent explicités les contraintes et exigences de rendus pour l'entreprise ainsi que les éventuels risques à rencontrer.\\

\subsubsection{Analyse des besoins}
\label{besoins}
Le besoin général était de réaliser un état de l'art des solutions de gestion des comptes à privilèges, de mettre en place 2 preuves de concept sur un environnement de test virtualisé afin de pouvoir définir le ou les solutions les plus adaptés à la gestion des comptes privilégiés. Ces solutions pouvaient (à l'époque de la réalisation de l'état de l'art) et sont (à ce jour) déployés chez un gros client. Je suis notamment intégré à l'équipe travaillant sur cette intégration dans l'infrastructure, car ayant réalisé une mise en place dans un environnement de test de la solution en question, je fais parti des personnes les plus compétentes de \textsc{Synetis} pour répondre aux différents problèmes qui pourraient se présenter.\\
Cet état de l'art devait aboutir à un document présentant le principe de gestion de comptes à privilèges, premièrement de façon théorique, puis de façon technique. Ensuite, ce document devait faire une analyse d'une liste de solutions d'éditeurs étant des acteurs majeurs sur le marché de la gestion de comptes à privilèges. Enfin, ce document a aura permis de créer un tableau comparatif de toutes les solutions étudiées, selon des critères pointus qui ont été définis comme répondant à la problématique du stage.

\subsubsection{Planning prévisionnel}
\label{planning}
Afin d'avoir une vue d'ensemble du projet, un planning a été réalisé à partir d'un découpage journalier des tâches à effectuer. Ce planning a permis d'avoir une vue marcoscopique du stage, et de répartir le travail sur la durée du stage, pour éviter d'avoir un retard qui pourrait surprendre à quelques semaines de la date buttoir. Ce découpage a aussi permis d'avoir des étapes, et de savoir à n'importe quel moment si j'avais de l'avance, ou surtout du retard, chose à proscrire pour arriver à un résultat convenable.\\
\begin{figure}[!ht]
    \center
    \includegraphics[width=\textwidth]{./images/calendrier_previsionnel.png}
    \caption{Calendrier prévisionnel}
\end{figure}

Bien sûr, le calendrier est une estimation et la réalité s'est avérée différente, ce point est abordé en section \ref{sec:resultats}. : la durée de recherche sur la gestion de comptes à privilèges et sur les différentes solutions a pris presque 2 fois plus de temps que prévu, tout comme le déploiement des solutions et le test de ces dernières. Cependant, le calendrier prévisionnel étant vu avec une large marge d'erreur, le stage tout de même pu être complété dans la durée impartie.

\subsection{Recherche}
\label{subsec:recherche}

La première étape a été la recherche d'informations sur le sujet. N'ayant pas de notions sur le sujet, j'ai d'abord commencé par me renseigner auprès des consultants de l'agence de Rennes, notamment mes tuteurs, Damien Seiler et Philippe Rolland, ainsi que le manager de l'agence, David Geffroy, ayant une base d'expertise dans le domaine. Grâce à ces premières lignes directrices fournis par ceux qui sont devenus mes collègues, j'ai pu orienter mes recherches internet vers la bonne direction, afin de trouver un maximum de résultats.

\subsubsection{Recherche du fonctionnement des solutions}
\label{par:fct_sol}
La meilleure façon de trouver des informations concernant le fonctionnement d'une solution de PAM s'est d'abord orientée vers la recherche d'informations génériques, comme des tutoriels ou des articles traitant du sujet. Cependant, j'ai fini par réaliser qu'il y avait très peu de ces ressources. La solution fut donc de directement s'orienter vers les solutions des éditeurs, et de tenter de comprendre leur fonctionnement, pour en tirer moi-même un fonctionnement général des solutions. Cette étape resta tout de même laborieuse, les éditeurs ne partageant pas énormément d'information quant à l'architecture ou le fonctionnement technique de leurs solutions, mais plutôt des caractéristiques de leur solution. Ceci ne m'empêcha pas de pouvoir trouver assez de solutions pour pouvoir en déduire une architecture assez claire, qui me donnait une vision d'ensemble du fonctionnement\footnote{Fonctionnement décrit dans un schéma à la section \ref{???}} d'une solution de PAM.\\
\textbf{INCLURE L'EXPLICATION DE FONCTIONNEMENT EN DÉTAIL ICI ?}

\subsubsection{Recherche des solutions existantes sur le marché}
\label{par:sol_market}
La recherche des solutions existantes sur le marché fut assez simple, compte tenu de la précédente recherche, s'appuyant sur ces solutions en question. Néanmoins, étant parti sur une base de 6 solutions trouvées pour réaliser un descriptif du fonctionnement général d'une solution de PAM, j'ai réussi à trouver plus du double de solutions par la suite, en navigant de lien en lien et en m'inscrivant à des newsletter m'envoyant des rapports tels que celui de l'éditeur \textsc{Forrester} écrit par Cser \cite{acs}.

\begin{figure}[!ht]
    \center
    \includegraphics[width=0.7\textwidth]{./images/forrester_quadrant.png}
    \caption{Quadrant mettant en évidence les acteurs du marché de PAM selon le rapport offre/stratégie et l'indice de présence sur le marché}
\end{figure}

Nous avons ainsi pu nous retrouver avec une liste de solutions satisfaisante pour pouvoir commencer à faire un comparatif *réaliste* (terme à revoir). Nous avons alors orienté mes recherches vers les spécificités des solutions, en parcourant toute la documentation disponible, en participant à des vision-conférences avec les commerciaux et ingénieurs des maisons d'édition ou en contactant le support. Cette étape a été celle qui a pris le plus de temps dans la période de recherche, qui parfois s'est avérée infructueuse au vu du manque d'informations disponibles et de l'absence de réponse du support (ou plus précisément des réponses me redirigeant vers des documents en ligne ne contenant pas les réponses demandées).
  \cleardoublepage
  \section{Résultats et discussion}
\label{sec:resultats}

\subsection{Des solutions adaptées à la taille des infrastructures}
\label{subsec:soltaille}


\subsection{L'avenir dans le cloud}
\label{subsec:avenircloud}

\subsection{CamStudio 2.7.4 : expérience personnelle d'attaque par escalade de privilèges}
\label{subsec:camstudio}
  \cleardoublepage
  \section*{Conclusion}
\addcontentsline{toc}{section}{Conclusion}

Une page tout au plus, résumé du travail accompli, faire apparaître si les objectifs ont été atteints, si de nouvelles difficultés ont été soulevées, propositions de solution et futur développement.

  \cleardoublepage
  \phantomsection\addcontentsline{toc}{section}{Références}
\begin{thebibliography}{ABC}
	\bibitem[1]{jpa} J-P. \textsc{Aumasson} \emph{Password Hashing: the Future is Now}, Blackhat US 2013, page 2, 2013.
    \bibitem[2]{rba} R.\textsc{Baloch} \emph{Ethical Hacking Penetration Testing Guide}, Auerbach Publications; 1 edition, 531 pages, 2014.
    \bibitem[3]{acs} Andreas \textsc{Cser} \emph{The Forrester Wave\texttrademark: Privileged Identity Management}, \textsc{Forrester$^{\mbox{\scriptsize{\textregistered}}}$}, Q3 2016.
    \bibitem[4]{gar} \textsc{Gartner}. Consulté le : 10/08/2016. \emph{Market Guide for Privileged Access Management}. Site web. \textsc{URL} : https://www.gartner.com/technology/media-products/newsletters/beyondtrust/1-3B8FB2Z/gartner.html
    \bibitem[5]{hma} \textsc{hMailServer}. Consulté le : 02/08/2016. \emph{Functionality - hMailServer - Free open source email server for Microsoft Windows}. Site web. \textsc{URL} : https://www.hmailserver.com/functionality
    \bibitem[6]{mic} \textsc{Microsoft Technet}. Consulté le : 18/08/2016. \emph{Privileged Access Management pour les services de domaine Active Directory}. Site web. \textsc{URL} : https://docs.microsoft.com/fr-fr/microsoft-identity-manager/pam/privileged-identity-management-for-active-directory-domain-services
    \bibitem[7]{min} \textsc{Minasi}, \textsc{Mark}, et al. \emph{Mastering Windows Server 2012 R2}, John Wiley \& Sons, 2013.
    \bibitem[8]{nist} \textsc{NIST} \emph{Best Practices for Privileged User PIV Authentication}, NIST Cybersecurity White Paper, 2016.
    \bibitem[9]{dsh} Dave \textsc{Shackleford} \emph{Keys to the Kingdom: Monitoring Privileged User Actions for Security and Compliance}, SANS Whitepaper, 2010.
    \bibitem[10]{skk} \textsc{Sundar, Koushicaa and Kumar, Sanjeev} \emph{Blue Screen of Death Observed for Microsoft Windows Server 2012 R2 under DDoS Security Attack}, Journal of Information Security, Volume 7, numéro 4, 2016.
    \bibitem[11]{jew} James E. \textsc{Weber}, Dennis \textsc{Guster}, Paul \textsc{Safonov} \& Mark B. \textsc{Schmidt}, \emph{Weak Password Security: An Empirical Study}, Information Security Journal: A Global Perspective, 17:1, 45-54, DOI:10.1080/10658980701824432, 2008.
\end{thebibliography}

  \cleardoublepage
  \section*{Annexe A}
\addcontentsline{toc}{section}{Annexe A}

\subsection{Fonctionnement détaillé des solutions de PAM}

Nous détaillerons dans cette annexe, le fonctionnement détaillé d'une solution de PAM. Afin d'illustrer les propos tenus, nous nous appuierons sur des schémas et digrammes de séquences.\\
Avant d'expliquer le fonctionnement d'une solution de PAM, nous allons voir comment les systèmes fonctionnent en temps normal.

\subsubsection{Sans solution de PAM}
\label{par:nopam}

Dans ce scénario, les utilisateurs finaux possèdent le mot de passe d'accès direct à la ressource protégée, comme on peut le voir sur la \textsc{Figure} \ref{fig:sans_PAM}.

\begin{figure}[!ht]
    \center
    \includegraphics[width=\textwidth]{./images/Schema_ultra_light_sans_PAM.png}
    \caption{Connexion à une ressource protégée sans solution de PAM}
    \label{fig:sans_PAM}
\end{figure}

On remarque qu'ici, les utilisateurs accèdent directement aux ressources protégées avec des mots de passe dédiés à chaque service/application/matériel. Il est donc très fréquent car inévitable que des utilisateurs partagent le même mot de passe. De plus, n'ayant aucune fédération (pas de supervision, ni de traçage), l'utilisateur est apte à effacer ses traces, par exemple en vidant les logs des applications et de ses actions\footnote{Par exemple sous Debian 8, la commande \texttt{cat /dev/null > ~/.bash\_history \&\& history -c \&\& exit} supprime toute action effectuée sur la machine avec le compte courant.}. On ne peut donc pas savoir ce que l'utilisateur a fait, ni quel utilisateur a fait des actions sur la ressource cible.\\
Le diagramme de séquence en \textsc{Figure} \ref{fig:diagseq_sans_PAM} décrit en détail les action qui se déroulent lors d'une telle connexion. Il permet de ausi de mettre en évidence l'absence de contrôle des utilisateurs : aucun registre d'évènements ne prend en compte les actions des utilisateurs. Les utilisaterus se partageant le même mot de passe pour chaque ressource cible, le contrôle n'est ainsi plus mis sur les utilisateurs, mais sur les ressources cibles, ce qui est l'opposé de ce que nous cherchons à faire. La \testsc{Figure} \ref{fig:invcont} schématise cette inversion du contrôle. 

\begin{figure}[!ht]
    \center
    \includegraphics[width=\textwidth]{./images/Sequence_noPAM_use.png}
    \caption{Diagramme de séquence détaillant les actions effectuées lors d'une connexion à une ressource sans solution de PAM}
    \label{fig:diagseq_sans_PAM}
\end{figure}

\begin{figure}[!ht]
    \center
    \includegraphics[width=\textwidth]{./images/ressource_centered.png}
    \caption{Schéma mettant en évidence l'inversion du contrôle de sécurité, mis sur les ressources plutôt que sur les utilisateurs}
    \label{fig:invcont}
\end{figure}

\subsubsection{Avec solution de PAM}
\label{par:withpam}

Dans ce scénario, une solution de PAM est en place. Ainsi les utilisateurs n'ont pas d'accès direct aux ressources cible. Toute l'architecture s'articule autour d'un composant central : le contrôleur d'accès appelé \textbf{bastion}. Tout accès à une ressource cible se fait via ce bastion. Cette architecture centralisée est schématisée dans la \textsc{Figure} \ref{fig:schempam}. Nous allons reprendre point par point un scénarion de connexion réussie à une ressource cible (en correspondance avec les étapes de la \textsc{Figure} \ref{fig:schempam}).

\begin{figure}[!ht]
    \center
    \includegraphics[width=\textwidth]{./images/Schema_PAM.png}
    \caption{Schéma décrivant l'architecture d'une solution de PAM intégrée dans une infrastructure}
    \label{fig:schempam}
\end{figure}

\begin{enumerate}
	\item L'utilisateur se connecte au bastion avec ses crédentiels
\end{enumerate}
  \cleardoublepage
  %\section*{Résumé} % Pas de numérotation

\textbf{Résumé}  La gestion de comptes à privilèges (PAM\footnote{Privileged Access Management}) est une sous-section de la gestion d’identité et d’accès (IAM\footnote{Identity and Access Management}). L’IAM est un large champ de contrôle d’accès qui se veut critique dans le domaine des technologies de l’information.\\
Il existe bien sûr une multitude de connexions spécifiques entre les utilisateurs et les dépendances technologiques. La PAM n’est que l’une d’entre elles.\\
La PAM est apparue au début des années 2000 à cause de l’impossibilité des solutions d’IAM de contrôler, gérer et faire des rapports sur les accès aux serveurs, aux bases de données, aux équipements réseau et tout autre application critique au sein d’une organisation. Cette solution entraîne une gestion d’un petit nombre d’utilisateurs, mais d’un grand nombre de dépendances technologiques ayant une importance clef dans le fonctionnement des infrastructures.
  \cleardoublepage
  %\section*{Résumé} % Pas de numérotation

\paragraph{Abstract}
The management of privileged accounts is a key area of access and identity management. It can track and log activity of accounts with elevated privileges such as \texttt{root} on \textsc{Linux/UNIX} or \textsc{Windows} \texttt{Administrator}. This allows to recover a misconfiguration that caused a disruption of services, prevent intrusions by escalating privileges on the system, and monitor potential service providers in a large groups (outsourcing of maintenance service). Commercial solutions offer different approaches to the problem of privileged accounts. This internship has been the subject of a study of these different solutions, their functionalities and operation. This study allowed us to highlight three products: \textsc{Wallix AdminBastion}, \textsc{CyberArk Privileged Access Security Solution} and \textsc{Thycotic Secret Server}, to finally deploy in 2 proof of concept. This development has allowed us to go further in understanding the operation of the management of privileged accounts, treaties points and points requiring additional treatment to lower risk on privileges accounts.
\end{document}

