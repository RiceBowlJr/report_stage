%\section*{Résumé} % Pas de numérotation

\paragraph{Résumé}
La gestion des comptes à privilèges est un domaine clef de la gestion d'accès et d'identité. Elle permet de suivre et journaliser les activités des comptes ayant des droits élevés comme \texttt{root} sur \textsc{Linux/UNIX} ou \texttt{Administrateur} sur \textsc{Windows}. Cette gestion permet ainsi de retrouver une erreur de configuration ayant entrainé une perturbation des services, de prévenir les intrusions par escalade de privilèges sur les système et de suivre d'éventuels prestataires de service dans un grand groupe (sous-traitance de la maintenance d'un service). Les solutions commerciales offrent différentes approches de la problématique des comptes à privilèges. Ce stage a donc fait l'objet d'une étude de ces différentes solutions, de leurs fonctionnalités et de leur fonctionnement. Cette étude a permis de faire ressortir 3 produits, \textsc{Wallix AdminBastion}, \textsc{CyberArk Privileged Access Security Solution} et \textsc{Thycotic Secret Server}, pour finalement en déployer 2 dans un environnement de test virtualisé. Cette mise en situation nous a permis d'aller plus en profondeur dans la compréhension du fonctionnement de la gestion des comptes à privilèges, des points traités et des points nécessitant des traitement supplémentaires à la diminution des risques des comptes à privilèges.

%La gestion de comptes à privilèges (PAM\footnote{Privileged Access Management}) est une sous-section de la gestion d’identité et d’accès (IAM\footnote{Identity and Access Management}). L’IAM est un large champ de contrôle d’accès qui se veut critique dans le domaine des technologies de l’information.\\
%Il existe bien sûr une multitude de connexions spécifiques entre les utilisateurs et les dépendances technologiques. La PAM n’est que l’une d’entre elles.\\
%La PAM est apparue au début des années 2000 à cause de l’impossibilité des solutions d’IAM de contrôler, gérer et faire des rapports sur les accès aux serveurs, aux bases de données, aux équipements réseau et tout autre application critique au sein d’une organisation. Cette solution entraîne une gestion d’un petit nombre d’utilisateurs, mais d’un grand nombre de dépendances technologiques ayant une importance clef dans le fonctionnement des infrastructures.