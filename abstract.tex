%\section*{Résumé} % Pas de numérotation

\paragraph{Abstract}  La gestion de comptes à privilèges (PAM\footnote{Privileged Access Management}) est une sous-section de la gestion d’identité et d’accès (IAM\footnote{Identity and Access Management}). L’IAM est un large champ de contrôle d’accès qui se veut critique dans le domaine des technologies de l’information.\\
Il existe bien sûr une multitude de connexions spécifiques entre les utilisateurs et les dépendances technologiques. La PAM n’est que l’une d’entre elles.\\
La PAM est apparue au début des années 2000 à cause de l’impossibilité des solutions d’IAM de contrôler, gérer et faire des rapports sur les accès aux serveurs, aux bases de données, aux équipements réseau et tout autre application critique au sein d’une organisation. Cette solution entraîne une gestion d’un petit nombre d’utilisateurs, mais d’un grand nombre de dépendances technologiques ayant une importance clef dans le fonctionnement des infrastructures.