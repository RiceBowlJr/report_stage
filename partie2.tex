\section{Méthodes et moyens mis en œuvre}
\label{sec:meth_moy}

Comme le titre l'indique, nous allons expliquer les méthodes et les moyens utilisés pour répondre à la problématique posée. Cette section sera séparée en 3 sous-sections : la première décrivant mon cheminement en matière de gestion de projet, la deuxième traitant de la phase de recherche d'informations, avec ses problèmes rencontrés et solutions trouvée, puis une troisième partie décrivant le travail effectué pendant la phase de test des deux produits sur un environnement de test, ainsi que toutes les différentes technologies utilisées.

\subsection{Gestion de projet}
\label{subsec:gestion}

La première chose que j'ai dû faire en arrivant dans l'entreprise, fut ce qu'on appelle chez \textsc{Synetis} une note de cadrage. Cette note de cadrage correspond, par rapport à ce qu'on a pu faire à l'université durant divers projets, à l'analyse des besoins et les résultats attendus, l'organisation en tâches de l'intégrité du stage. Cette répartition des tâches dans le temps a permis de scinder l'ensemble du projet en de multiples étapes simples, courte, qui m'a permis de segmenter mon stage pour ne pas me retrouver perdu ou submergé par le travail. En dernière partie furent explicités les contraintes et exigences de rendus pour l'entreprise ainsi que les éventuels risques à rencontrer.\\

\subsubsection{Analyse des besoins}
\label{besoins}
Le besoin général était de réaliser un état de l'art des solutions de gestion des comptes à privilèges, de mettre en place 2 preuves de concept sur un environnement de test virtualisé afin de pouvoir définir le ou les solutions les plus adaptés à la gestion des comptes privilégiés. Ces solutions pouvaient (à l'époque de la réalisation de l'état de l'art) et sont (à ce jour) déployés chez un gros client. Je suis notamment intégré à l'équipe travaillant sur cette intégration dans l'infrastructure, car ayant réalisé une mise en place dans un environnement de test de la solution en question, je fais parti des personnes les plus compétentes de \textsc{Synetis} pour répondre aux différents problèmes qui pourraient se présenter.\\
Cet état de l'art devait aboutir à un document présentant le principe de gestion de comptes à privilèges, premièrement de façon théorique, puis de façon technique. Ensuite, ce document devait faire une analyse d'une liste de solutions d'éditeurs étant des acteurs majeurs sur le marché de la gestion de comptes à privilèges. Enfin, ce document a aura permis de créer un tableau comparatif de toutes les solutions étudiées, selon des critères pointus qui ont été définis comme répondant à la problématique du stage.

\subsubsection{Planning prévisionnel}
\label{planning}
Afin d'avoir une vue d'ensemble du projet, un planning a été réalisé à partir d'un découpage journalier des tâches à effectuer. Ce planning a permis d'avoir une vue marcoscopique du stage, et de répartir le travail sur la durée du stage, pour éviter d'avoir un retard qui pourrait surprendre à quelques semaines de la date buttoir. Ce découpage a aussi permis d'avoir des étapes, et de savoir à n'importe quel moment si j'avais de l'avance, ou surtout du retard, chose à proscrire pour arriver à un résultat convenable.\\
\begin{figure}[!ht]
    \center
    \includegraphics[width=\textwidth]{./images/calendrier_previsionnel.png}
    \caption{Calendrier prévisionnel}
\end{figure}

Bien sûr, le calendrier est une estimation et la réalité s'est avérée différente, ce point est abordé en section \ref{sec:resultats}. : la durée de recherche sur la gestion de comptes à privilèges et sur les différentes solutions a pris presque 2 fois plus de temps que prévu, tout comme le déploiement des solutions et le test de ces dernières. Cependant, le calendrier prévisionnel étant vu avec une large marge d'erreur, le stage tout de même pu être complété dans la durée impartie.

\subsection{Recherche}
\label{subsec:recherche}

La première étape a été la recherche d'informations sur le sujet. N'ayant pas de notions sur le sujet, j'ai d'abord commencé par me renseigner auprès des consultants de l'agence de Rennes, notamment mes tuteurs, Damien Seiler et Philippe Rolland, ainsi que le manager de l'agence, David Geffroy, ayant une base d'expertise dans le domaine. Grâce à ces premières lignes directrices fournis par ceux qui sont devenus mes collègues, j'ai pu orienter mes recherches internet vers la bonne direction, afin de trouver un maximum de résultats.

\subsubsection{Recherche du fonctionnement des solutions}
\label{par:fct_sol}
La meilleure façon de trouver des informations concernant le fonctionnement d'une solution de PAM s'est d'abord orientée vers la recherche d'informations génériques, comme des tutoriels ou des articles traitant du sujet. Cependant, j'ai fini par réaliser qu'il y avait très peu de ces ressources. La solution fut donc de directement s'orienter vers les solutions des éditeurs, et de tenter de comprendre leur fonctionnement, pour en tirer moi-même un fonctionnement général des solutions. Cette étape resta tout de même laborieuse, les éditeurs ne partageant pas énormément d'information quant à l'architecture ou le fonctionnement technique de leurs solutions, mais plutôt des caractéristiques de leur solution. Ceci ne m'empêcha pas de pouvoir trouver assez de solutions pour pouvoir en déduire une architecture assez claire, qui me donnait une vision d'ensemble du fonctionnement\footnote{Fonctionnement décrit dans un schéma à la section \ref{???}} d'une solution de PAM.\\
\textbf{INCLURE L'EXPLICATION DE FONCTIONNEMENT EN DÉTAIL ICI ?}

\subsubsection{Recherche des solutions existantes sur le marché}
\label{par:sol_market}
La recherche des solutions existantes sur le marché fut assez simple, compte tenu de la précédente recherche, s'appuyant sur ces solutions en question. Néanmoins, étant parti sur une base de 6 solutions trouvées pour réaliser un descriptif du fonctionnement général d'une solution de PAM, j'ai réussi à trouver plus du double de solutions par la suite, en navigant de lien en lien et en m'inscrivant à des newsletter m'envoyant des rapports tels que celui de l'éditeur \textsc{Forrester} écrit par Cser \cite{acs}.

\begin{figure}[!ht]
    \center
    \includegraphics[width=0.7\textwidth]{./images/forrester_quadrant.png}
    \caption{Quadrant mettant en évidence les acteurs du marché de PAM selon le rapport offre/stratégie et l'indice de présence sur le marché}
\end{figure}

Nous avons ainsi pu nous retrouver avec une liste de solutions satisfaisante pour pouvoir commencer à faire un comparatif *réaliste* (terme à revoir). Nous avons alors orienté mes recherches vers les spécificités des solutions, en parcourant toute la documentation disponible, en participant à des vision-conférences avec les commerciaux et ingénieurs des maisons d'édition ou en contactant le support. Cette étape a été celle qui a pris le plus de temps dans la période de recherche, qui parfois s'est avérée infructueuse au vu du manque d'informations disponibles et de l'absence de réponse du support (ou plus précisément des réponses me redirigeant vers des documents en ligne ne contenant pas les réponses demandées). C'est par ailleurs une des étapes qui a complètement décalé le calendrier prévisionnel, prenant sur la marge prévue à cet effet.\\
Cette étape a conduit à éditer un tableau comparatif des solutions, dont nous pouvons trouver un aperçu à la \textsc{Figure }\ref{fig:tabcomp}. Ce tableau comparatif n'est pas disponible en annexe, à cause de sa taille impossible à imprimer, mais il reste tout de même disponible sur demande en format \emph{Microsoft Excel}.

\begin{figure}[!ht]
    \center
    \includegraphics[width=\textwidth]{./images/tabcomp.png}
    \label{fig:tabcomp}
    \caption{Extrait du tableau comparatif de solutions édité au terme de la phase de recherche}
\end{figure}

\subsection{Proof of Concept}
\label{subec:poc}

La phase de conception d'environnement de test s'est déroulée en plusieurs étapes :
\begin{itemize}
	\item Conception architecture idéale
	\item Validation de l'architecture avec le tuteur, et remaniement de cette dernière pour s'adapter aux ressources matériels disponibles chez \textsc{Synetis}, ressources assez limitées car nous étions en saturation de ressources mémoire et processeur sur le serveur interne. Un nouveau serveur est arrivé en fin de stage, mais malheureusement quelques semaines trop tard
	\item Installation de l'infrastructure et déploiement des solutions à tester
\end{itemize}

\subsubsection{Architecture}
\label{par:archi}

L'architecture idéale sur laquelle les solutions devaient être intégrée nous semblait être celle qui se rapprochait le plus d'une situation réelle d'entreprise, comprenant des séparations logiques pour les différents corps de métier (

En effet, les ressources limitées ont réduit notre infrastructure de test au strict minimum, donc une machine de chaque type :
\begin{itemize}
	\item \textsc{MS Windows Server 2012 R2} : serveur de test de connexion RDP\footnote{Remote Desktop Protocol : contrôle à distance d'un serveur.}, de configuration des solutions de PAM (base de données \emph{MySQL} et \emph{SQL Server}), de mail (avec le logiciel \emph{hMailServer} \cite{hma} et contrôleur de domaine \emph{Active Directory}
	\item \textsc{MS Windows Server 2012} : serveur TSE\footnote{Terminal SErvice : permet d'utiliser le serveur pour faire tourner des applications utilisées en bureau distant.} permettant de tester une connexion sur une application distante (\emph{VMWare vSphere Client}\footnote{Programme Windows permettant de configurer un hôte de virtualisation et de faire tourner des machines virtuelles.} dans notre cas)
	\item \textsc{Linux Debian 8.4.0 Jessie} : serveur Linux permettant de tester une connexion SSH\footnote{Secure SHell : protocole de connexion à distance à une machine Linux/Unix.}
\end{itemize}

Cette architecture limitée nous a permis de tester et d'éprouver 2 des solutions choisies, tout en respectant les contraintes de ressources établies.\\
Bien plus que les solutions en elles-même, le déploiement de l'infrastructure de base a nécessité d'autres technologies, que l'ont peut lister en tâches suivantes :
\begin{itemize}
	\item Sur Microsoft Windows Server 2012 et 2012 R2 :
		\begin{arrowlist}
			\item Installation du rôle contrôleur de domaine\footnote{gestionnaire d'un domaine sous le système d'exploitation Windows.}
			\item Création et configuration d'un domaine, d'une forêt et de toutes ses dépendances assurant le bon fonctionnement de l'infrastructure
			\item Création d'utilisateurs de domaine\footnote{Utilisateurs créés dans un annuaire Active Directory, disponible pour toute machine du domaine.} avec les droits nécessaires et suffisant à leur fonctionnement, grâce aux groupes de sécurité Windows (consulter le livre de Minasi \emph{et al.} \cite{min} pour plus d'informations sur Active Directory et la sécurité de Windows Server 2012 R2)
			\item Création de comptes de service de domaine (MSA\footnote{Managed Service Account : compte Active Directory dédié aux service, le système gère lui-même les mots de passe.}) sous \emph{Powershell}\footnote{Interface en ligne de commande et langage de scripting dédié à Windows.}
			\item Installation de système de gestion de base de données relationnelle \emph{MySQL} et \emph{SqlServer} et création de bases de données pour les solutions de PAM
			\item Installation et configuration d'un serveur de mail local \emph{hMailServer} pour récupérer les mails envoyés par les solutions de PAM
			\item Installation du rôle TSE et configuration de ce dernier avec l'application \emph{VMWare vSphere Client}
		\end{arrowlist}
	\item Sur Debian 8.4.0 :
		\begin{arrowlist}
			\item Installation du serveur SSH
			\item Configuration réseau
		\end{arrowlist}
\end{itemize}

\subsubsection{Choix des solutions de PAM à tester}
\label{par:choixsol}

Nous avions fait, au terme de la phase de recherche, une sélection de 3 potentielles solutions à déployer sur nos environnements de test. Ce choix s'était fait en prenant en compte les données présentées dans le tableaux comparatif des solutions dont on a un aperçu dans la \textsc{Figure} \ref{fig:tabcomp}. Ces 3 solutions étaient :
\begin{itemize}
	\item \textsc{Wallix AdminBastion}
	\item \textsc{CyberArk Privileged Account Security Suite}
	\item \textsc{Thycotic SecretServer}
\end{itemize}

Nous étions déjà en contact avec \emph{Wallix}, car partenaires et mon tuteur était en train de passer une formation avec eux, pour la solution en question. Nous avons donc pu avoir facilement une installation de leur solution. En revanche, étant partenaire de \emph{Wallix}, \emph{CyberArk} a refusé de nous fournir une version d'évaluation tant que nous n'abandonnions pas notre partenariat, afin qu'il devienne notre partenaire exclusif. Ce choix de \emph{CyberArk} venant du fait que \emph{Wallix} est leur plus gros concurrent en France, car \emph{Wallix} est une entreprise Française et que beaucoup d'entreprises jouent le jeu de faire fonctionner une entreprise locale\footnote{Information fournie par le troisième éditeur, \emph{Thycotic} durant des échanges de mails.}. Nous avons donc déclineé la solution de \emph{CyberArk} et sommes entrés en contact avec \emph{Thycotic}, avec qui nous n'avons pas eu de soucis et qui nous ont offert un suivi remarquable (une communication omniprésente à toutes les étapes de test).

\subsubsection{Déploiement : Wallix AdminBastion}
\label{par:wallix}

La version d'essai de \emph{Wallix AdminBastion} se présente sous forme d'une machine virtuelle (fichier \texttt{vmdk}\footnote{VMWare Virtual Disk : format de disque virtuel créé par \emph{VMWare}}).