\section{Résultats et discussion}
\label{sec:resultats}

Dans cette partie, nous allons discuter des résultats obtenus, pour déterminer s'il y a une solution plus à même de répondre aux besoins d'une entreprise avec laquelle \textsc{Synetis} pourrait envisager travailler.

\subsection{Des solutions se démarquant par des spécificités}
\label{subsec:soltaille}

Après avoir travaillé avec les deux différentes solutions (\emph{Wallix} et \emph{Thycotic}), il s'est clairement imposé que ces solutions apportent une approche similaire de la gestion des comptes à privilèges. Cette ressemblance d'utilisation a aussi été pointée lorsque je réalisais les PoC par l'étude de marché produit par l'entreprise de conseil Gartner\footnote{Gartner est une entreprise américaine de conseil et de recherche fondée en 1979. Elle vend des recherches et des analyses dans le domaine des technologies de l'information.} \cite{gar}. Les seuls gros points notables les différenciant la haute disponibilité qui n'est pas présente chez \emph{Thycotic}, la gestion des interactions d'application à application absente chez \emph{Wallix} et les formats de licences ne se recoupant que sur la disponibilité en mode \og cloud \fg{}.\\
Nous pouvons aussi noter une différence majeure entre ces deux solutions, au niveau des choix technologiques : \emph{Wallix} a choisi de développer sa solution sous Linux, tandis que \emph{Thycotic} s'appuie sur un système Windows Server. Ceci implique une facilité d'installation non-négligeable pour \emph{Wallix}, ne nécessitant pas de configuration préalable car l'ensemble du moteur du bastion est contenu dans la machine virtuelle (ou serveur physique, selon la licence choisie). Cette solution est complète, pour ainsi dire \og prête à l'emploi \fg{} car il ne reste que la configuration des ressources et utilisateurs à gérer. A l'opposé, \emph{Thycotic} fourni un exécutable à installer sur un serveur Windows, avec plusieurs solutions lourdes\footnote{Le serveur Windows d'une part, puis Microsoft SQL Server, Microsoft Internet Information Services et .NET Framework pour faire tourner l'interface de la solution.} à installer au préalable. Cette configuration n'a en l'occurrence pas fonctionné dans notre cas avec un serveur de test, il nous a fallu reprendre une installation \gls{stratch}.\\
Il faut aussi ajouter à cela que l'ensemble de la sécurité de \emph{Secret Server} s'appuie donc sur la sécurité du système d'exploitation Windows, étant le plus touché par les attaques et donc le plus apte à avoir des failles de sécurité mises à jour, comme l'on démontré Sundar \textsc{et coll.}\cite{skk} sur une faiblesse de Windows Server 2012 R2 (soit la dernière version de Windows Server) face à une attaque de type DDoS (Syn flood\footnote{Bombardement de requête TCP SYN}).
%Expliquer pourquoi c'est de la merde Windows

\subsection{L'avenir dans le cloud}
\label{subsec:avenircloud}

Selon l'étude de Gartner\cite{gar}, les besoins de solution de \gls{pam} dans le cloud représentent seulement 3\% des parts de marché, pour selon leur estimation, 30\% en 2019. Ces chiffres en sont pas surprenants compte tenu de l'évolution des infrastructures décentralisées, comme le proposent les grands éditeurs comme Microsoft avec la suite \emph{365}, le géant du net Google avec son \emph{Google Enterprise}, Amazon avec l'ensemble de ses solutions qui vont du stockage de données (\emph{Amazon EC2}) au cluster de calcul (\emph{Amazon S3}).\\
Il est important de noter que cette gestion des comptes à privilèges dans le cloud représente un certain défis, sachant que ces infrastructures ne sont pas maintenues par les entreprises concernées (client final) mais par ces grands éditeurs. On peut alors se poser des questions sur la responsabilité des comptes à privilèges : relève-t-elle du client ou de l'éditeur ?

\subsection{Une synergie entre solution de PAM et une fédération d'identité}
\label{subsec:syner}

Un des leaders du marché des solutions de \gls{pam}, \emph{Centrify} (voir l'étude de marché menée par \textsc{Cser} \cite{acs} pour plus d'informations), propose un produit apportant une synergie entre la gestion des comptes à privilèges et la fédération d'identité\footnote{Gestion des accès et supervision des utilisateurs communs, par exemple la gestion des comptes des étudiants à l'université.}. En effet, il semble logique d'intégrer une solution de gestion des comptes à privilèges avec une solution de fédération d'identités, la gestion des utilisateurs en sera facilitée, et l’homogénéité du système offrira une assurance de qualité de service. Cependant, \emph{Centrify} est pour le moment le seul à proposer une telle solution, avec Microsoft et sa solution MIM (Microsoft Identity Manager) qui inclut l'extension PIM (Privileged Identity Manager) gérant les comptes à privilèges avec des groupes Active Directory de sécurité à usage limité dans le temps (\emph{shadow group}, voir l'article de \textsc{Microsoft Technet} \cite{mic} pour pus d'informations).

\subsection{CamStudio 2.7.4 : expérience personnelle d'attaque par escalade de privilèges}
\label{subsec:camstudio}

Le poste de travail qui m'a été fourni durant ce stage était un ordinateur de dépannage servant pour les nouveaux arrivants à \textsc{Synetis} Rennes. Au bout de quelques mois de stage, j'ai eu une notification douteuse me demandant de mettre à jour un logiciel (Yahoo! Search) qui n'étais pas installé sur le poste. Avec l'aide de plusieurs collègues, nous avons donc cherché à comprendre d'où venait cette notification, ce qui l'exécutait et quelles actions elle générait.\\
Nous avons commencé par suivre le processus avec le gestionnaire de tâches de Windows. Le processus était déclenché par un fichier présent dans le répertoire \texttt{AppData}, qui est le répertoire de stockages de données des applications s'exécutant sur Windows. Le fichier ayant lancé le processus (fichier avec une extension .dat) n'existait plus, mais il restait cependant un fichier de logs conséquent (plus de 1000 lignes de logs). Ce fichier nous a donné des informations sur les tentatives d'exécution de code sur la machine, en l'occurrence beaucoup d'ouvertures de shell, infructueuses avec le compte local. Un tel accès pourrait mener à une compromission complète du poste de travail, pouvant se traduire par un destruction du système, ou pire encore, un vol de données. Le vol de données serait le pire des scénario, si l'on tient compte que \textsc{Synetis} est un sous-traitant/intégrateur de solution de fédération d'identités pour des grands groupes tels que EDF ou encore des conseils régionaux, détenant ainsi les accès à l'administration de leur infrastructure.\\
Par la suite, nous avons trouvé comment était exécuté la fenêtre qui apparaissait : c'était une ancienne technologie, un fichier hta exécuté par un moteur de Windows. Ce moteur permet de convertir n'importe quel fichier binaire en page html dans une fenêtre embarquée. Nous avons réussi à trouver où menait le lien présent sur cette fenêtre : vers deux domaines qui ne répondaient pas lorsque l'on essayait de les joindre via un navigateur web. Ceci était sûrement dû à des paramètres traitant la requête sur le serveur n'acceptant que un certain type de requête, par exemple un \texttt{user-agent}\footnote{Dans une requête HTML, l'\texttt{user-agent} est le type de client lançant la requête, dans notre cas, notre navigateur web, par exemple Mozilla Firefox.} spécifique.\\
Après une recherche sur le net, nous avons trouvé un sujet ouvert sur le forum \emph{Reddit} discutant de ce malware, qui s'avère être présent dans une installation du logiciel CamStudio 2.7.4 compromis.\\
Finalement, la désinstallation de ce logiciel a empêché le malware de nuire. Toutefois, c'est une situation dans laquelle une solution de \gls{pam} aurait bloqué le problème à la source, en limitant l'installation d'un tel logiciel, si l'on considère que même l'installation de logiciel sur un poste est géré par les administrateurs de l'entreprise.

\subsection{Les bonnes pratiques à intégrer}
\label{subsec:pratiques}

Même avec une infrastructure sécurisée au plus proche de la perfection, les plus grosses failles restent l'humain. Il est donc important, lors d'un déploiement de solution de \gls{pam} de sensibiliser le personnel à certaines bonnes pratiques de sécurité incontournables. Ces pratiques sont détaillées dans une publication faite par le \textsc{NIST}\footnote{National Institute of Standards and Technology : institut national des standards et de la technologie des USA.} \cite{nist}.

\subsubsection{Renforcer l'authentification}
\label{par:auth}

Comme il a été vu en introduction, l'utilisation de mot de passe faible, devinable par ingénierie sociale\footnote{Deviner le mot de passe d'une personne en se renseignant sur ses habitudes, sa vie et ses relations sociales.}, d'algorithme de hachage faible, ou un faible chiffrement d'un mot de passe qui serait recouvrable par une \emph{rainbow table}\footnote{Table de correspondance de mots et de leur hash, disponible et utilisable en ligne.}.\\
Pour éviter de rencontrer des infiltrations par ces failles, il est important de mettre en place une politique de renforcement de l'authentification avec :
\begin{itemize}
	\item Une authentification forte : multi-facteur, au minimum double-facteur, par exemple l'utilisation du couple login/mot de passe et d'une validation avec envoi de SMS sur le téléphone personnel de l'employé. Plusieurs solutions déjà implémentés sont utilisables, comme par exemple \emph{Google Authenticator} ou le MFA de \emph{Microsoft}. Il est aussi possible de mettre en place un système PIV (Personnal Identity Verification, souvent une carte à puce à insérer dans un lecteur).
	\item Une augmentation des couches de sécurité dans le chiffrement des mots de passe : utilisation d'un algorithme de hachage fort comme le SHA-256 ou AES-256 doublé d'un salage (comme l'explique durant la conférence Blackhat US 2013 \textsc{Aumasson} \cite{jpa}) de ce hash avec une chaîne aléatoire.
	\item La mise en place d'une politique de mot de passe forte qui force l'utilisateur à utiliser un mot de passe d'une longueur minimale, avec un nombre minimal de caractères spéciaux, lettres minuscules et majuscules et chiffres. Il est même recommandé d'utiliser un passphrase, qui augmente la taille de la chaîne de caractères, en remplaçant les lettres par des chiffres et des symboles\footnote{Par exemple "4" pour "A", "\$" pour "s", etc.}.
\end{itemize}

\subsubsection{Minimiser les accès privilégiés}
\label{par:minipriv}

Le but de ce stage étant la gestion des comptes à privilèges, nous avons clairement pu comprendre que le meilleur moyen de limiter les compromissions de système était de limiter au maximum les accès privilégiés. La pratique est donc de commencer par :
\begin{itemize}
	\item Supprimer les accès des comptes à privilèges qui ne nécessitent plus ce type d'accès (par exemple l'administration de système, réseau ou base de données qui sont des tâches ponctuelles).
	\item Supprimer ou désactiver tous les comptes à privilèges qui ne sont plus nécessaires (y compris les comptes natifs du système).
	\item Supprimer l'excès de privilèges d'un compte en tenant compte du contexte de l'entreprise, plus que celui applicatif.
	\item Supprimer toutes les permissions inutiles des comptes à privilèges, notamment les commandes non-relative à l'utilisation de ces comptes.
	\item Réduire au minimum la durée d'une session privilégiée unique.
	\item Exiger une re-connexion lorsque cette dernière a expiré.
\end{itemize}

