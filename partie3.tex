\section{Résultats et discussion}
\label{sec:resultats}

Dans cette partie, nous allons discuter des résultats obtenus, pour déterminer s'il y a une solution plus à même de répondre aux besoins d'une entreprise avec laquelle \textsc{Synetis} pourrait envisager travailler.

\subsection{Des solutions se démarquant par des spécificités}
\label{subsec:soltaille}

Après avoir travaillé avec les deux différentes solutions (\emph{Wallix} et \emph{Thycotic}), il s'est clairement imposé que ces solutions apportent une approche similaire de la gestion des comptes à privilèges. Cette ressemblance d'utilisation a aussi été pointée lorsque je réalisais les PoC par l'étude de marché produit par l'entreprise de conseil Gartner\footnote{Gartner est une entreprise américaine de conseil et de recherche fondée en 1979. Elle vend des recherches et des analyses dans le domaine des technologies de l'information.} \cite{gar}. Les seuls gros points notables les différenciant la haute disponibilité qui n'est pas présente chez \emph{Thycotic}, la gestion des interactions d'application à application absente chez \emph{Wallix} et les formats de licences ne se recoupant que sur la disponibilité en mode \og cloud \fg{}.\\
Nous pouvons aussi noter une différence majeure entre ces deux solutions, au niveau des choix technologiques : \emph{Wallix} a choisi de développer sa solution sous Linux, tandis que \emph{Thycotic} s'appuie sur un système Windows Server. Ceci implique une facilité d'installation non-négligeable pour \emph{Wallix}, ne nécessitant pas de configuration préalable car l'ensemble du moteur du bastion est contenu dans la machine virtuelle (ou serveur physique, selon la licence choisie). Cette solution est complète, pour ainsi dire \og prête à l'emploi \fg{} car il ne reste que la configuration des ressources et utilisateurs à gérer. A l'opposé, \emph{Thycotic} fourni un exécutable à installer sur un serveur Windows, avec plusieurs solutions lourdes\footnote{Le serveur Windows d'une part, puis Microsoft SQL Server, Microsoft Internet Information Services et .NET Framework pour faire tourner l'interface de la solution.} à installer au préalable. Cette configuration n'a en l'occurrence pas fonctionné dans notre cas avec un serveur de test, il nous a fallu reprendre une installation \gls{stratch}.\\
Il faut aussi ajouter à cela que l'ensemble de la sécurité de \emph{Secret Server} s'appuie donc sur la sécurité du système d'exploitation Windows, étant le plus touché par les attaques et donc le plus apte à avoir des failles de sécurité mises à jour, comme l'on démontré Sundar \textsc{et coll.}\cite{skk} sur une faiblesse de Windows Server 2012 R2 (soit la dernière version de Windows Server) face à une attaque de type DDoS (Syn flood\footnote{Bombardement de requête TCP SYN}).
%Expliquer pourquoi c'est de la merde Windows

\subsection{L'avenir dans le cloud}
\label{subsec:avenircloud}

\subsection{Une synergie entre solution de PAM et une fédération d'identité}
\label{subsec:syner}

Exemple fédé : Ping Identity https://www.pingidentity.com/fr.html

\subsection{CamStudio 2.7.4 : expérience personnelle d'attaque par escalade de privilèges}
\label{subsec:camstudio}

\subsection{Les bonnes pratiques à intégrer}
\label{subsec:pratiques}

Même avec une infrastructure sécurisée au plus proche de la perfection, les plus grosses failles restent l'humain. Il est donc important, lors d'un déploiement de solution de \gls{pam} de sensibiliser le personnel à certaines bonnes pratiques de sécurité incontournables. Ces pratiques sont détaillées dans une publication faite par le \textsc{NIST}\footnote{National Institute of Standards and Technology : institut national des standards et de la technologie des USA.} \cite{nist}.

\subsubsection{Renforcer l'authentification}
\label{par:auth}

Comme il a été vu en introduction, l'utilisation de mot de passe faible, devinable par ingénierie sociale\footnote{Deviner le mot de passe d'une personne en se renseignant sur ses habitudes, sa vie et ses relations sociales.}, d'algorithme de hachage faible, ou un faible chiffrement d'un mot de passe qui serait recouvrable par une \emph{rainbow table}\footnote{Table de correspondance de mots et de leur hash, disponible et utilisable en ligne.}.\\
Pour éviter de rencontrer des infiltrations par ces failles, il est important de mettre en place une politique de renforcement de l'authentification avec :
\begin{itemize}
	\item Une authentification forte : multi-facteur, au minimum double-facteur, par exemple l'utilisation du couple login/mot de passe et d'une validation avec envoi de SMS sur le téléphone personnel de l'employé. Plusieurs solutions déjà implémentés sont utilisables, comme par exemple \emph{Google Authenticator} ou le MFA de \emph{Microsoft}. Il est aussi possible de mettre en place un système PIV (Personnal Identity Verification, souvent une carte à puce à insérer dans un lecteur).
	\item Une augmentation des couches de sécurité dans le chiffrement des mots de passe : utilisation d'un algorithme de hachage fort comme le SHA-256 ou AES-256 doublé d'un salage (comme l'explique durant la conférence Blackhat US 2013 \textsc{Aumasson} \cite{jpa}) de ce hash avec une chaîne aléatoire.
	\item La mise en place d'une politique de mot de passe forte qui force l'utilisateur à utiliser un mot de passe d'une longueur minimale, avec un nombre minimal de caractères spéciaux, lettres minuscules et majuscules et chiffres. Il est même recommandé d'utiliser un passphrase, qui augmente la taille de la chaîne de caractères, en remplaçant les lettres par des chiffres et des symboles\footnote{Par exemple "4" pour "A", "\$" pour "s", etc.}.
\end{itemize}

\subsubsection{Minimiser les accès privilégiés}
\label{par:minipriv}

Le but de ce stage étant la gestion des comptes à privilèges, nous avons clairement pu comprendre que le meilleur moyen de limiter les compromissions de système était de limiter au maximum les accès privilégiés. La pratique est donc de commencer par :
\begin{itemize}
	\item Supprimer les accès des comptes à privilèges qui ne nécessitent plus ce type d'accès (par exemple l'administration de système, réseau ou base de données qui sont des tâches ponctuelles).
	\item Supprimer ou désactiver tous les comptes à privilèges qui ne sont plus nécessaires (y compris les comptes natifs du système).
	\item Supprimer l'excès de privilèges d'un compte en tenant compte du contexte de l'entreprise, plus que celui applicatif.
	\item Supprimer toutes les permissions inutiles des comptes à privilèges, notamment les commandes non-relative à l'utilisation de ces comptes.
	\item Réduire au minimum la durée d'une session privilégiée unique.
	\item Exiger une re-connexion lorsque cette dernière a expiré.
\end{itemize}

