\section*{Introduction} % Pas de numérotation
\label{intro}
\addcontentsline{toc}{section}{Introduction} % Ajout dans la table des matières

%-> Environnement professionnel et l'entreprise. <-
%
%-> Objectifs du travail personnel et moyens mis en oeuvre pour les atteindre <-
%
%-> Présentation claire du plan adopté poru la suite du corps du mémoire <-
%
%Pas plus de 2 pages.
Le choix de mon entreprise de stage s'est orienté vers \textsc{Synetis}, qui est une porte ouverte vers le monde de la sécurité, dans lequel je souhaite m'orienter malgré ma formation qui n'est pas spécialisée dans ce domaine.\\
\textsc{Synetis} est une PME\footnote{Petite à Moyenne Entreprise} à taille humaine basée dans le centre de Paris. Une agence a été ouverte à Rennes en 2012 lors du recrutement de spécialistes en gestion des identités sur cette zone. Cette agence de Rennes compte 6 consultants et 2 managers (dont le chef d'agence). L'ambiance est très conviviale, tout le monde se connaît, des évènements de cohésion sont régulièrement organisés (afterwork, repas en groupe le vendredi midi). De même, une réunion trimestrielle est organisée sur le siège de Paris, qui regroupe tout le personnel de \textsc{Synetis}, afin de faire un bilan des projets, de rencontrer les nouveaux arrivants, et de partager les nouveaux objectifs visés par l'entreprise.\\
Le travail est réparti par équipe sur les différents projets en cours, chaque consultant pouvant travailler sur plusieurs projets en même temps. On peut distinguer deux types de projet : les gros projets s'étalant sur de longues périodes (plus de six mois) et les projets courts durant au maximum quatre mois. L'entreprise est spécialisée dans la gestion d'identité pour les grands groupes (conseils régionaux, grandes entreprises nationales et internationales), mais commence à développer une branche de pentest (test d'intrusion sur différentes structures, pour de plus amples informations, le guide de Rafay Baloch \cite{rba} est très complet).\\
J'étais intégré comme tous les autres consultants, avec un tuteur travaillant sur des projets correspondants au sujet de mon stage qui me guidait au travers de mes recherches et développements.\\
Ce stage concerne les systèmes de gestion de comptes à privilèges, qui sont les comptes avec des droits élevés, comme ceux des compte \textsc{root} des systèmes d'exploitation Linux/UNIX et \textsc{Administrateur} sur Windows. Durant ce stage, il était question de comprendre ne profondeur les enjeux de la gestion de comptes à privilèges, les moyens existant pour établir cette gestion pour pouvoir mieux comparer l'ensemble des solutions disponibles sur le marché. Cette comparaison a permis de considérer 2 solutions à mettre en œuvre sur un environnement virtuel pour comprendre ces systèmes dans les moindres détails, et en faire une comparaison encore plus poussée. Toutes les informations récoltées nous auront finalement permis de comprendre comment les solutions répondent à la problématique de la gestion des comptes à privilèges, et de déterminer quelles solutions répondent le mieux à cette problématique, afin de pouvoir proposer de l'intégration avec de futurs clients.

Le plan adopté pour la suite de ce mémoire consiste en une première section décrivant la problématique à résoudre, et les origines de cette problématique. Une deuxième section expliquera les moyens et les méthodes mises en œuvre pour répondre à cette problématique et enfin une dernière partie expliquera les résultats obtenus.