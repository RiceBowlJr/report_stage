\section*{Introduction} \label{intro}% Pas de numérotation
\addcontentsline{toc}{section}{Introduction} % Ajout dans la table des matières

%-> Environnement professionnel et l'entreprise. <-
%
%-> Objectifs du travail personnel et moyens mis en oeuvre pour les atteindre <-
%
%-> Présentation claire du plan adopté poru la suite du corps du mémoire <-
%
%Pas plus de 2 pages.
Tout d'abord, mon choix de stage s'est orienté vers l'entreprise \textsc{Synetis}, car c'est une porte ouverte vers le monde de la sécurité, dans lequel je souhaite m'orienter malgré ma formation qui n'est pas spécialisée dans ce domaine.\\
\textsc{Synetis} est une PME\footnote{Petite à Moyenne Entreprise} à taille humaine basée dans le centre de Paris. Une agence a été ouverte à Rennes en 2012 lors du recrutement de spécialistes en gestion des identités sur cette zone. Cette agence de Rennes compte 6 consultants et 2 managers (dont le chef d'agence). L'ambiance est très conviviale, tout le monde se connaît, des évènements de cohésion sont régulièrement organisés (afterwork, repas en groupe le vendredi midi). De même, une réunion trimestrielle est organisée sur le siège de Paris, qui regroupe tout le personnel de \textsc{Synetis}, afin de faire un bilan des projets, de rencontrer les nouveaux arrivants, et de partager les nouveaux objectifs visés par l'entreprise.\\
Le travail est réparti par équipe sur les différents projets en cours, chaque consultant pouvant travailler sur plusieurs projets en même temps. On peut distinguer deux types de projet : les gros projets s'étalant sur de longues périodes (plus de six mois) et les projets courts durant au maximum quatre mois. L'entreprise est spécialisée dans la gestion d'identité pour les grands groupes (conseils régionaux, grandes entreprises nationales et internationales), mais commence à développer une branche de pentest (test d'intrusion sur différentes structures, pour de plus amples informations, le guide de Rafay Baloch \cite{rba} est très complet).\\
J'étais intégré comme tous les autres consultants, avec un tuteur travaillant sur des projets correspondants au sujet de mon stage qui me guidait au travers de mes recherches et développements.\\
La gestion de comptes à privilèges (PAM\footnote{Privileged Access Management}) est une sous-section de la gestion d’identité et d’accès (IAM\footnote{Identity and Access Management}). L’IAM est un large champ de contrôle d’accès qui se veut critique dans le domaine des technologies de l’information.\\
Il existe bien sûr une multitude de connexions spécifiques entre les utilisateurs et les dépendances technologiques. La PAM n’est que l’une d’entre elles.\\
La PAM est apparue au début des années 2000 à cause de l’impossibilité des solutions d’IAM de contrôler, gérer et faire des rapports sur les accès aux serveurs, aux bases de données, aux équipements réseau et tout autre application critique au sein d’une organisation. Cette solution entraîne une gestion d’un petit nombre d’utilisateurs, mais d’un grand nombre de dépendances technologiques ayant une importance clef dans le fonctionnement des infrastructures.\\
La fonctionnalité principale d’une solution de gestion d’identité privilégiée gravite autour de la sécurisation de l’accès aux ressources critiques par les administrateurs IT.\\
Plus précisément, des privilèges spécifiques peuvent être délivrés à des comptes d’utilisateur. Ces privilèges sont dépendants des systèmes et des applications impliqués, mais ils peuvent inclure la capacité à écrire des données, créer des comptes, exécuter une mission, pour ne citer que quelques exemples. En plus de contrôler l’accès avec un grand niveau de finesse, beaucoup de solutions fournissent aussi un package d’actions disponibles lors de l’utilisation d’un compte privilégié. Dans le but d’assurer la sécurité de ces comptes, les solutions exploitent un large éventail de mécanismes, comme par exemple, l’utilisation de clefs SSH, la rotation de mot de passe et l’ajout de multiples authentifications (authentification multi-facteur).\\
La gestion d’identités privilégiées est devenue de plus en plus importante, notamment avec la croissance des requis de sécurité et des règles de conformité (l’ISO 27001 qui est une norme de sécurité international sur la protection des actifs informationnels par exemple). Il est aussi important de noter que de nombreux exploits\footnote{Element permettant d'exploiter une faille de sécurité} compromettant des données sont liés à la compromission d’accréditations privilégiées. Avec la recrudescence des tentatives de hack et les contrecoups économiques de plus en plus conséquents, les entreprises commencent à apporter de plus en plus d’intérêt à la gestion et au contrôle des comptes à privilèges. En effet, les tentatives de piratage sont de plus en plus variées : du social engineering, au vol de ces accréditations par une brèche dans la sécurité en passant par des \textit{brute force} de mot de passe n’ayant pas une complexité suffisante\footnote{L'utilisation de mots de passe faibles tels que \texttt{password}, \texttt{admin}, \texttt{1234}, \texttt{azerty1234} ou \texttt{Abcd1234} reste encore très fréquente. Pour trouver ces mots de passe faibles, la technique la plus employée est le brute force avec un dictionnaire de mots de passe}.\\
L'objectif de ce stage était donc de réaliser une étude sur le fonctionnement des solutions PAM, puis d'en sélectionner deux qui répondraient aux besoins exprimés en terme de sécurité et de facilité de mise en place. Ces deux solutions ont par la suite été testées dans un PoC\footnote{Proof of Concept : preuve de concept, est une démonstration de faisabilité, ici sur un environnement de simulation de réseau d'entreprise générique} afin d'en comprendre au mieux le fonctionnement, de les éprouver et détecter les éventuelles caractéristiques supplémentaires à ajouter dans l'état de l'art.

Le plan adopté pour la suite de ce mémoire consiste en une première section décrivant la problématique à résoudre, et les origines de cette problématique. Une deuxième section expliquera les moyens et les méthodes mises en œuvre pour répondre à cette problématique et enfin une dernière partie expliquera les résultats obtenus.