%\section{Section}
%
%\subsection{Sous section}
%
%\subsubsection{Sous sous section}
%
%\paragraph{Paragraphe}
%
%Pas plus bas dans la hiérarchie. Minimum 2 éléments par division hiérarchique.
%
%Equilibrer la taille de chaque chapitre (à l'exception de l'introduction et de la conclusion).
%
%Exemple de développement :
%- Problématique
%- Matériels et méthodes
%- Résultats et discussion

%\begin{figure}[!ht]
%    \center
%    \includegraphics[width=0.5\textwidth]{./images/logo-univ.jpg}
%    \caption{logo université, échelle 50\% de la page}
%\end{figure}
\section{Problématique : comment pallier le manque de contrôle des comptes privilégiés}

L'objectif général est de résoudre le problème de manque de contrôle et de monitoring des comptes à privilèges. Afin de parvenir à ce résultat, plusieurs solutions ont été développées par des éditeurs, le but final de ce projet étant de déterminer quelle est, ou quelles sont, la ou les meilleures solutions permettant de résoudre ce problème.\\
Pour répondre à cette problématique, nous allons d'abord expliquer plus en détails ce qu'est un compte à privilèges, puis nous intéresser aux raisons pour lesquelles ces comptes sont des points clefs de la sécurité des systèmes d'information. Ensuite, nous allons expliquer un fonctionnement global et commun aux solutions du marché puis mettre en avant les limitations que peuvent avoir ces solutions de \gls{pam}.

\subsection{Contexte}

\subsubsection{Les comptes à privilèges}
Il semble important de définir clairement la différence entre un compte à privilèges et un compte d'utilisateur classique, et plus précisément les deux catégories de mot de passe qu'ils engendrent :
\begin{description}
	\item [Les mots de passe utilisateur] : basiquement, un mot de passe est un secret qui permet l'utilisation d'un compte. Un compte représente un utilisateur humain et son mot de passe justifie son identité, comme un compte Active Directory\footnote{Système d'annuaire électronique propriétaire de \textsc{Microsoft}, basé sur la norme LDAP (Lightweight Directory Access Protocol).}, qui représente digitalement un humain, et le mot de passe qui justifie l'identité de l'humain qui s'y connecte auprès du système. Ce type de mot de passe est connu par l'utilisateur humain qui est représenté par le compte, le but étant d'avoir un minimum de comptes par entité humaine, l'idéal étant une correspondance bijective\footnote{Un seul utilisateur humain pour un seul compte et vice-versa.}
	\item [Les mots de passe de comptes à privilèges] : ces mots de passe sont des mots de passe liés à un compte qui ne représente pas une entité humaine. Ce compte peut être un compte système comme \texttt{root} sur un système d'exploitation \textsc{Linux} ou un compte de service sur un système d'exploitation \textsc{Windows}. Ces mots de passe ne sont pas forcément fournis à un utilisateurs humain, et même idéalement, ne doivent pas l'être, ainsi, ils peuvent d'être d'une complexité élevée sans avoir un soucis d'apprentissage de ce dernier
\end{description}

Le but de ce stage était de trouver la meilleure solution pour sécuriser les mots de passe de comptes à privilèges tout en supervisant les comptes liés à des utilisateurs.


\subsubsection{La gestion de comptes à privilèges}
La gestion de comptes à privilèges (PAM\footnote{Privileged Access Management}) est une sous-section de la gestion d’identité et d’accès (IAM\footnote{Identity and Access Management}). L’IAM est un large champ de contrôle d’accès qui se veut critique dans le domaine des technologies de l’information.\\
Il existe bien sûr une multitude de connexions spécifiques entre les utilisateurs et les dépendances technologiques. La \gls{pam} n’est que l’une d’entre elles.\\
La \gls{pam} est apparue au début des années 2000 à cause de l’impossibilité des solutions d’IAM de contrôler, gérer et faire des rapports sur les accès aux serveurs, aux bases de données, aux équipements réseau et tout autre application critique au sein d’une organisation. Cette solution entraîne une gestion d’un petit nombre d’utilisateurs, mais d’un grand nombre de dépendances technologiques ayant une importance clef dans le fonctionnement des infrastructures.\\
La fonctionnalité principale d’une solution de gestion d’identité privilégiée gravite autour de la sécurisation de l’accès aux ressources critiques par les administrateurs IT.\\
Plus précisément, des privilèges spécifiques peuvent être délivrés à des comptes d’utilisateur. Ces privilèges sont dépendants des systèmes et des applications impliqués, mais ils peuvent inclure la capacité à écrire des données, créer des comptes, exécuter une mission, pour ne citer que quelques exemples. En plus de contrôler l’accès avec un grand niveau de finesse, beaucoup de solutions fournissent aussi un package d’actions disponibles lors de l’utilisation d’un compte privilégié. Dans le but d’assurer la sécurité de ces comptes, les solutions exploitent un large éventail de mécanismes, comme par exemple, l’utilisation de clefs SSH, la rotation de mot de passe et l’ajout de multiples authentifications (authentification multi-facteur).\\
La gestion d’identités privilégiées est devenue de plus en plus importante, notamment avec la croissance des requis de sécurité et des règles de conformité (l’ISO 27001 qui est une norme de sécurité international sur la protection des actifs informationnels par exemple). Il est aussi important de noter que de nombreux exploits\footnote{Element permettant d'exploiter une faille de sécurité} compromettant des données sont liés à la compromission d’accréditations privilégiées. Avec la recrudescence des tentatives de hack et les contrecoups économiques de plus en plus conséquents, les entreprises commencent à porter de plus en plus d’intérêt à la gestion et au contrôle des comptes à privilèges. En effet, les tentatives de piratage sont de plus en plus variées : du social engineering, au vol de ces accréditations par une brèche dans la sécurité en passant par des \textit{brute force} de mot de passe n’ayant pas une complexité suffisante\footnote{L'utilisation de mots de passe faibles tels que \texttt{password}, \texttt{admin}, \texttt{1234}, \texttt{azerty1234} ou \texttt{Abcd1234} reste encore très fréquente. Pour trouver ces mots de passe faibles, la technique la plus employée est le brute force avec un dictionnaire de mots de passe} (dont traitent Weber \emph{et coll.} \cite{jew}).\\

\subsection{Un point clef de la sécurité des systèmes d'information}

\subsubsection{Des prestataires de service ayant les clefs du royaume}
\label{par:presta_kingdom}

Le principal point de sécurité posant question avec les comptes à privilèges est leurs utilisateurs.\\
Dans une entreprise, le personnel gérant l'infrastructure n'est pas forcément d'une dimension adaptée aux besoins. Cette dimension n'est souvent pas accessible et est comblée par l'embauche de prestataires de service. On peut prendre l'exemple d'une entreprise de taille moyenne, qui gère son infrastructure en autonomie. Seulement, l'entreprise grossit et une nouvelle infrastructure déployée en \emph{clusters}\footnote{Grappe de serveurs sur le réseau, aussi appelé ferme de calcul.} est nécessaire. Le personnel n'ayant pas les compétences pour réaliser cette mise à niveau, l'entreprise devra faire appel à une entreprise de service tierce, qui réalisera et maintiendra cette nouvelle infrastructure. Ce sont les prestataires de service.\\
Cependant, ces prestataires doivent intervenir avec des droits élevés d'administration. Ils ont donc ce qu'on appelle les clefs du royaume\footnote{En anglais, littéralement \emph{keys to the kingdom}, représente l'accès sans limite à toute l'infrastructure informatique.}, sans aucune supervision.\\
Du point de vue de la sécurité, il est très risqué de recourir à de telles pratiques. C'est ici qu'une solution doit être trouvée, et c'est ici que les systèmes de gestion des comptes à privilèges ont un rôle clef.

\subsubsection{Une cible de choix pour les pirates}
Un compte privilégié est un compte utilisé par les administrateurs système et réseau, ainsi que par les équipes de sécurité pour accéder aux ressources réseau comme les serveurs, les pare-feux, les switches, les routeurs, les ordinateurs, les applications ou les bases de données avec des droits élevés. Ces comptes sont nécessaires à la maintenance d'une infrastructure, tout comme aux interventions de réparation, de diagnostique ou gestion de situations de crise\footnote{Attaque sur un serveur par exemple}. Dans de grands groupes, il peut y avoir un grand nombre de ces comptes, de l'ordre de la centaine voire du millier d'entités, répartis sur plusieurs sites.\\
Ces comptes peuvent aussi être des comptes d'application communicant avec d'autres applications\footnote{A2A : Application To Application, littéralement d'application à application}, comme par exemple un serveur faisant une sauvegarde régulière sur un autre serveur de récupération.\\
Tous ces comptes ne sont pas surveillés par les traditionnels gestionnaires d'identités, seul un mot de passe permet d'y accéder. De plus ces comptes sont très souvent des comptes partagés entre plusieurs administrateurs pour une question de facilité de gestion.\\
Ainsi, une personne mal intentionnée réussissant à voler les accès d'un tel compte verrait son pouvoir de destruction, voire de vol d'informations sans limites, ce qui en fait une cible privilégiée par les pirates informatiques.

\subsubsection{Un manque de visibilité d'actions} Comme abordé en introduction, il existe un grand manque de visibilité sur les actions des comptes à privilèges. En effet, ces comptes aux droits élevés, ne sont ni tracés, ni surveillés. Ceci peut avoir plusieurs mauvaises incidences, certaines intentionnelles et malveillantes, d'autre involontaires, mais tout de même paralysantes.\\
On peut classer ces incidences en deux catégories, l'une relevant d'une erreur accidentelle, et l'autre d'une volonté de nuire à une organisation comme le décrit l'article de \textsc{Shackleford} \cite{dsh} :
\begin{itemize}
	\item Erreur accidentelle :
	\begin{arrowlist}
		\item Erreur de configuration, difficile à retrouver à cause du manque de supervision
	\end{arrowlist}
	\item Volonté de nuire :
	\begin{arrowlist}
		\item Sabotage d'une configuration, menant à un déni de service
		\item Vol d'informations sensibles
	\end{arrowlist}
\end{itemize}

Ce manque de  visibilité crée aussi un point noir dans un audit de sécurité : il n'y a aucune détection de faille de sécurité concernant ces comptes.

\subsection{Objectifs d'un système de \gls{pam}}

D'après les points précédents, on peut en déduire les spécificités que nous voudrions améliorer vis-à-vis des comptes à privilèges :
\begin{itemize}
	\item Centraliser l’accès aux données de l’entreprise
 	\item Sécuriser les comptes à privilèges (duo identifiants et mot de passe)
 	\item Gérer de manière forte des mots de passe et établir une politique d’authentification forte
	\item Journaliser et superviser les activités des comptes à privilèges
\end{itemize}

\subsection{Fonctionnement d'un système de gestion de comptes à privilèges}

\subsubsection{Cas général}
Le principe commun à toutes les solutions des éditeurs est la présence d'une identification sur un serveur central. On peut considérer 2 groupes : les comptes à privilèges et les ressources à protéger. Entre ces 2 entités vient s'intercaler le serveur d'identification, qui fait la passerelle entre les comptes et les ressources.\\
L'identification d'un utilisateur sur un compte à privilèges se fait sur le serveur central, et l'identification de ce compte à privilèges sur une ressource protégée est déléguée au serveur central. Ainsi, les utilisateurs n'ont plus à gérer et connaître les mots de passe d'accès aux ressources protégées; c'est le serveur central qui détient tous les secrets\footnote{Les logins et mot de passe, aussi appelés \og \glspl{credential} \fg{} en anglais.}.\\
Le serveur central a, à sa charge, de protéger les mots de passe dans un coffre-fort et de les renouveler régulièrement (il est préconisé de renouveler ses mots de passe au moins une fois par mois, ce qui est déjà peu pour une entreprise).

\begin{figure}[!ht]
    \center
    \includegraphics[width=\textwidth]{./images/Schema_ultra_light_PAM.png}
    \caption{Fonctionnement général d'un système de \gls{pam}}
\end{figure}

\subsubsection{Deux types d'architecture}
Malgré le principe de fonctionnement identique pour quasiment toutes les solutions de \gls{pam}, il existe cependant des différences au niveau de l'architecture de ces dernières solutions, on distingue 2 grandes familles :
\begin{itemize}
	\item Architecture proxy\footnote{Toutes les communications transitent par un point de contrôle.} : les ressources et les comptes d'utilisateurs sont séparés physiquement (ou logiquement\footnote{Redirection des paquet  par port.}), le serveur central gère tout seul les accès aux applications
	\item Architecture avec agents : les accès aux ressources et la supervision sont gérés par des agents sur les ressources cibles (application installée sur la ressource)
\end{itemize}
Nous pouvons rapidement nous rendre compte qu'une architecture avec des agents est beaucoup plus intrusive et longue ou difficile à mettre en place qu'une architecture en proxy, où seul l'adressage est à modifier.

\subsection{Les limitations des solutions \gls{pam}}

\subsubsection{Le facteur humain} Il faut noter que même avec un système de \gls{pam} éprouvé et efficace, nous ne pouvons pas mettre de côté le risque le plus exploitable dans le domaine de la sécurité qu'est, le facteur humain. En effet, il est souvent plus facile de tromper un élément du personnel pour une compromission de données. Il sera donc important lors d'un déploiement d'une solution de \gls{pam}, d'éduquer le personnel de l'organisation concernée, afin que ces derniers mettent correctement en œuvre les règles de base de la sécurité informatique comme par exemple :
\begin{itemize}
	\item Totalement prohiber \og l'effet \textit{post-it} \fg{} : notation de mot de passe sur un \textit{post-it} collé sur l'écran
	\item Toujours remplacer les mots de passe d'usine dans les logiciels utilisés
	\item Forcer le changement régulier (au minimum une fois par mois) de mot de passe des utilisateurs
\end{itemize}

\subsubsection{La détection de comportements anormaux}
Très peu de solutions de \gls{pam} proposent un système de détection de comportements anormaux, comme par exemple un conseiller commercial qui aurait accès aux informations des comptes clients, qui abuserait de ce droit.\\
En effet, même avec une restriction des droits, une supervision et journalisation des activités, une solution de \gls{pam} n'est pas une intelligence artificielle qui peut détecter des comportements suspects. Cependant, il est toujours possible de détecter des comportements prédéfinis avec des successions de commandes qui lèveraient une alerte.

%L’objectif est de mettre en place un système de contrôle des risques liés aux comptes à privilèges, donc tout d’abord avoir un socle central d’accès. Ce socle doit constituer l’élément central du système d'information par lequel circulent tous les flux venant des utilisateurs vers les données de l’entreprise. Ainsi, ce socle gère tous les accès aux ressources protégées, et les utilisateurs n’ont plus à connaître les mots de passe des ressources, mais juste leur mot de passe personnel pour se connecter au socle principal.\\
%Le choix d’une solution doit prendre en compte l’environnement technique, les habitudes des administrateurs et les logiciels utilisés par l’entreprise.\\
%Afin de parvenir à un système sûr et performant, nous pouvons dégager les objectifs suivants :
%\begin{itemize}
%	\item Centraliser l’accès aux données de l’entreprise.
%	\item Sécuriser les comptes à privilèges.
%	\item Gérer de manière forte des mots de passe et établir une politique d’authentification forte.
%	\item Journaliser les activités des comptes à privilèges.
%	\item Automatiser les processus pour améliorer la productivité tout en renforçant la sécurité.
%\end{itemize}